\section{使用说明}

在 \code{src/} 目录下运行:
\begin{lstlisting}
pip install -r requirements.txt
py app.py
\end{lstlisting}
并在浏览器中访问 Flask 所显示的 URL(默认 \url{http://127.0.0.1:5000/})。唯一的用户名为 \code{tianzq20},密码为 \code{password}。

\section{代码结构}

此文档中以下的路径都默认添加 \code{src/} 前缀。

本项目的 HTML、CSS、Javascript、以及图片文件分别存放在 \code{templates}、\code{static/css}、\code{static/js}、以及 \code{static/img}。实现后端的 Flask 程序存放在 \code{app.py} 中。

\section{实现}

\subsection{内容}

此项目中有两个网页:“登录”页面,即让用户输入用户名以及密码,以及“成功”页面,即用户成功地登录后可以看到自己的连接时常以及已用流量。它们分别存放在 \code{templates/login.html} 和 \code{templates/success.html} 中。后者与实验一相比几乎相同。

\subsection{样式}

用于为 \code{login.html} 以及 \code{success.html} 添加样式的 CSS 文件分别存放在 \code{static/css/login.css} 以及 \code{static/css/succeess.css} 中。

\subsection{行为}

后端是此项目的重点之一,而它的实现在 \code{app.py} 中使用了 Flask 框架。本项目有两个接口,其 URL 路径分别在 \code{/login}(或 \code{/})以及 \code{/success},也有两个状态:连接状态 \code{connected} 以及已用流量 \code{usage}。在未连接状态下,用户访问 \code{/success} 后将会被重定向到 \code{/login}。\code{/login} 在接收 \code{GET} 请求后就会反应登录页面的 HTML 代码。在接收 \code{POST} 请求,即用户名以及密码,就会根据正确性而返回对应的响应。\code{POST} 请求的发送是由 \code{/static/js/login.js} 中的 \code{connect()} 函数实现的。此函数也负责实现空用户名或空密码的异常处理,以及将用户名和密码以 JSON 格式发送给后端。登录成功后重定向到 \code{/success} 也是 \code{login.js} 中的 \code{connect()} 函数实现的。

类似的,在已连接状态下,用户访问 \code{/login} 后将会被重定向到 \code{/success}。\code{/success} 在接收 \code{GET} 请求后就会反应成功页面的 HTML 代码。在接收 \code{POST} 请求,即用户点击“断开链接”,就会重定向到 \code{/login},并将已用流量增加。

\section{问题与解决方法}

一个我遇到的问题是:已用流量不仅需要通过文字在成功页面上显示,而且也需要改变此页面上的橘色进度条的长度。Flask 所使用的模板可以让后端动态地改变 HTML 中的内容,但较难改变样式,因此无法改变进度条的长度。一个方式是让后端将已用流量传输给 Javascript,并让 Javascript 改变文字以及进度条。比如:后端实现另外一个接口,只返回已用流量。Javascript 就可以发出请求,并从回应中得到已用流量。但是会到一定的程度增加代码量。此项目的解决方法是将已用流量的文字通过模板放在 HTML 中(使用 \code{usage} 变量),并将 Javascript 读取 HTML 文件中的元素,从而的去已用流量的数值,最后可以改变进度条的长度。