\section{使用说明}

使用浏览器打开 \code{ext/index.html}。加载其中的 Vue.js 脚本需要链接互联网。

\section{代码结构}

本实验的代码存放在 \code{ext/index.html}、\code{ext/style.css}、\code{ext/index.js} 中,实现了内容、样式、和行的分离。所有被使用的图片存放在 \code{/ext/imgs} 中。

\section{实现}

\subsection{内容}

以下介绍已实现的内容以及实现方法:

\begin{itemize}
  \item 整个 Vue.js 应用在 \code{id="login-success} 的 \code{div} 中存放
  \item 对话框:由一个正方形的 \code{div}、一个三角形的 \code{div}、以及文字组成
  \item 信息框:
  \begin{itemize}
    \item 用户名
    \item 已链接:表示用户链接的时常,由时:分:秒的格式显示
    \item 已用流量:通过文字以及进度条显示
  \end{itemize}
  \item “断开链接”按钮:白色的 \code{div} 中放置灰色的按钮,使用 \code{div} 显示灰色三角形
  \item 导航:四个链接由文字以及图标组成的链接
\end{itemize}

\subsection{样式}

为了与原网页更加类似,许多属性,包括各个元素的大小以及字体,是通过浏览源代码而发现并使用的。

\newpage

\subsubsection{颜色}

此项目通过浏览原网页的源代码,或者截图使用图像编辑器寻找每个元素的颜色。考虑到了以下元素的颜色:

\begin{itemize}
  \item 页面背景
  \item 橘色对话框以及其中的“欢迎”文字
  \item 用户名
  \item 已间接时间
  \item 灰色的已用流量刻度文字以及片段
  \item 橘色的已用流量进度条
  \item “断开连接”在非悬停、悬停、点击三个状态
  \item 链接在非悬停、悬停、点击三个状态
\end{itemize}

\subsubsection{图片}

原网页中使用了多张图片包括:

\begin{itemize}
  \item “欢迎” 文字
  \item “断开连接” 按钮
  \item 导航链接图标
  \item 网页图标
\end{itemize}

都是使用图片而显示的。前面两个图像很容易使用纯 HTML + CSS 实现,并且显示效果更好,因此此项目只使用了导航链接和网页图标这两种图片。

\newpage

\subsection{行为}

\begin{itemize}
  \item “断开链接”按钮
  \begin{itemize}
    \item 当光标悬停在按钮上,按钮的背景颜色会变化,光标从箭头变成手
    \item 当光标离开按钮时,按钮的背景颜色以及光标的形状将会复原
  \end{itemize}
  \item 导航
  \begin{itemize}
    \item 可以通过点击文字,也可以通过点击图标访问对应的网站
    \item 当光标悬停在链接上,文字的颜色不变,下划线以及链接指向的网站的中文名和其 URL 将出现
    \item 点击链接后,它指向网页将在新的分页中加载,原始网页不被改变
  \end{itemize}
\end{itemize}

\subsubsection{Vue.js}

此网页包括若干个将会动态变化的元素,包括:

\begin{itemize}
  \item 用户名
  \item 以链接时间
  \item 已用流量的数值以及进度条的长度
\end{itemize}

以上的值没有写死在 HTML 代码中,而是使用 Vue.js 的 \code{data} 函数传递给 HTML。由于实验一只要求静态的网页,这些值在 \code{index.js} 中的代码写死了,但是以后可以容易地实现它们的动态变化。

此外,存在多个导航链接,它们的 HTML 代码的重复性较高,而且以后有可能增加更多链接,则链接是使用 \code{Vue.js} 动态创建的。