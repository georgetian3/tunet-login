\section{使用说明}

在 \code{src/} 目录下运行:
\begin{lstlisting}
pip install -r requirements.txt
py app.py
\end{lstlisting}

并在浏览器中访问 Flask 所显示的 URL(默认 \url{http://127.0.0.1:5000/})。唯一的用户名为 \code{tianzq20},密码为 \code{password}。

\section{代码结构}

此文档中以下的路径都默认添加 \code{ext/} 前缀。

此实验的代码结构与实验二完全一样,即 HTML、CSS、Javascript、以及图片文件分别存放在 \code{templates}、\code{static/css}、\code{static/js}、以及 \code{static/img}。实现后端的 Flask 程序存放在 \code{app.py} 中。与实验二不同之处是所有使用的 Javascript 库文件都存放在 \code{static/js} 下,包括 \code{vue.js} 以及 \code{echarts.js},则不联网也可以运行此实验。

\section{实现}

此实验在实验二的基础上添加了显示流量使用情况的动态折线图。使用的是 Apache ECharts 库,并且参考了以下来自官网的例子:\url{https://echarts.apache.org/examples/en/editor.html?c=dynamic-data2}。在 \code{templates/success.html} 中的底部添加了 \code{id="chart"} 的 \code{div},用于包含图表,并在 \code{static/css/success.css} 中设置了它的大小以及位置。

主要的代码添加到了 \code{static/js/static.js} 的尾端。其中定义了两个全局变量: \code{data} 用于存放图表中显示的十个数据点,而 \code{usage} 记录总共流量使用。调用的 \code{setInterval} 函数将每两秒往 \code{data} 中存数据,并只保留数组中的后十个元素。每个数据点是由当前的时间以及当前总共使用的流量组成。为了让图表中的圆圈与横轴对齐,目前的时间的毫秒部分被设为零。每次产生一个数据点也会将 \code{usage} 增加零至一之间的随机的浮点数。当鼠标悬停在图表上,提示将会出现,其格式为 “\code{HH}:\code{mm}:\code{ss} \code{x}.\code{yz}GB”。

\section{问题}

当网页长期在后台运行,折线图上的圆圈可能会集中在图的右端。此问题较难复制,目前没有找到解决方法。